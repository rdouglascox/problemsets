\documentclass{article}\usepackage{amssymb}\usepackage[utf8]{inputenc}\usepackage{qtree}\usepackage[width=18.00000cm]{geometry}\title{Problem Set 02: PL Tables (Answers)}\author{}\date{}\begin{document}\maketitle{}\begin{enumerate}\item Use a truth table to test whether the following propositions are equivalient.\begin{enumerate}\item $\lnot{}$(L$\rightarrow{}$L), ((L$\wedge{}$L)$\wedge{}$$\lnot{}$L)\item Answer:\begin{quote}\begin{tabular}{|c||c|c|}\hline L&$\lnot{}$(L$\rightarrow{}$L)&((L$\wedge{}$L)$\wedge{}$$\lnot{}$L)\tabularnewline \hline T&F&F\tabularnewline F&F&F\tabularnewline \hline \end{tabular}\end{quote}\end{enumerate}\item Use a truth table to test whether the following argument is valid.\begin{enumerate}\item ((Y$\rightarrow{}$Y)$\wedge{}$Y), ((Y$\vee{}$Y)$\rightarrow{}$(Y$\rightarrow{}$Z)) $\therefore{}$(Z$\vee{}$$\lnot{}$X)\item Answer:\begin{quote}\begin{tabular}{|c|c|c||c|c|c|}\hline X&Y&Z&((Y$\rightarrow{}$Y)$\wedge{}$Y)&((Y$\vee{}$Y)$\rightarrow{}$(Y$\rightarrow{}$Z))&(Z$\vee{}$$\lnot{}$X)\tabularnewline \hline T&T&T&T&T&T\tabularnewline T&T&F&T&F&F\tabularnewline T&F&T&F&T&T\tabularnewline T&F&F&F&T&F\tabularnewline F&T&T&T&T&T\tabularnewline F&T&F&T&F&T\tabularnewline F&F&T&F&T&T\tabularnewline F&F&F&F&T&T\tabularnewline \hline \end{tabular}\end{quote}\end{enumerate}\end{enumerate}\end{document}